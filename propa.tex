\documentclass[12pt, a4paper]{article}

\usepackage[spanish]{babel}
\usepackage{amsmath}
\usepackage{amsfonts}
\usepackage{cancel}
\usepackage[labelfont=bf]{caption}
\usepackage{hyperref}
\hypersetup{
    colorlinks,
    citecolor={red!50!black},
    linkcolor={blue!50!black},
    urlcolor={blue!80!black}
}


\usepackage{mathtools}
\usepackage{multicol}
% \usepackage{esvect}

\usepackage{physics}
\DeclareDocumentCommand\vnabla{}{\nabla}
\usepackage{siunitx}
\AtBeginDocument{\RenewCommandCopy{\qty}{\SI}}
\DeclareSIUnit \rayl {rayl}

\usepackage{parskip}
\usepackage{esint}
% \usepackage{geometry}
\usepackage{xcolor}

% Vectors
% \renewcommand{\vec}[1]{\mathbf{#1}}
\makeatletter
\newcommand{\vv}[2][]{
    \@ifempty{#1}{\vec{#2}}{\vec{#2}_{#1}}
}
\makeatother

\title{Apuntes de Propagación de Ondas}
\author{Javier Rodrigo López}
\date{\today}
 

%%%%%%%%%%%%%%%%%%%%%%%%%%%%%%%%%%%%%%%%%%%%%%%%%%%%%
\begin{document}
% \renewcommand{\arraystretch}{1.2}
\maketitle

% Table of contents
\tableofcontents

\newpage
\section*{Introducción}

El primer parcial incluye los temas 1 a 4. El segundo parcial incluye los temas 5 a 9. Se debe obtener una nota mínima de 3 en cada uno y la media debe ser mayor o igual a 5. Ambos parciales ponderan un 50\% en la nota final. Se puede optar a evaluación final en lugar del segundo parcial.

\subsection*{Fechas de examen}

\begin{description}
    \item [Primer parcial] 13 de noviembre
    \item [Segundo parcial o global] 20 de enero
\end{description}

\subsection*{Ejercicios propuestos}
Hay que hacer los ejercicios propuestos antes de que el profesor los resuelva en clase. Los que serán resueltos en clase son los siguientes:
\begin{multicols}{5}
    \begin{itemize}
        \item 1.3
        \item 1.6
        \item 2.1
        \item 2.4
        \item 3.5
        \item 4.3
        \item 4.5
        \item 5.3
        \item 5.7
        \item 6.3
        \item 6.7
        \item 7.3
        \item 7.5
        \item 8.2
        \item 8.4
        \item 9.2
        \item 9.10
    \end{itemize}
\end{multicols}

\subsection*{Tareas pendientes}

\begin{itemize}
    \item Relación entre vectores unitarios \textbf{sin mezclar coordenadas}. Es decir, un ejemplo:
    \[ \frac{x}{\sqrt{x^2 + y^2}} \qquad \text{en lugar de} \qquad \cos \varphi \]
\end{itemize}

\newpage
\section{Herramientas: operadores vectoriales}

Una \textbf{función de onda} es una función en el espacio y el tiempo que describe el comportamiento de una onda.

El sistema de referencia será siempre un triedro orientado a derechas, de modo que se cumpla:
\begin{equation} \label{eq:triedro}
    \begin{split}
        \vv[x]{u} + \vv[1]{x} \cp \vv[y]{u} = + \vv[z]{u} \\
        \vv[y]{u} \cp \vv[x]{u} = - \vv[z]{u} 
    \end{split}
\end{equation}

Las \textbf{coordenadas} de un punto representan la posición de ese punto en el espacio.

Una \textbf{superficie coordenada} es una superficie que se define por una ecuación de la forma $f(x, y, z) = \text{cte.}$

El vector posición de un punto $P$ en el espacio se denota como $\vv{r}$, y se puede expresar en función de las coordenadas del punto (diferente en cada sistema de coordenadas).

\subsection{Sistemas de coordenadas}
\subsubsection{Coordenadas cartesianas}

Recomendables para problemas con \textbf{simetría plana}. Los valores que pueden tomar las coordenadas son:
\begin{equation}
    \begin{aligned}
        x \in (-\infty, +\infty) \\
        y \in (-\infty, +\infty) \\
        z \in (-\infty, +\infty)
    \end{aligned}
\end{equation}

\begin{itemize}
    \item La coordenada $x$ es la distancia del punto a la superficie $yz$.
    \item La coordenada $y$ es la distancia del punto a la superficie $xz$.
    \item La coordenada $z$ es la distancia del punto a la superficie $xy$.
\end{itemize}

Cualquier vector puede expresarse como combinación lineal de los vectores unitarios.En cualquier sistema de coordenadas, los vectores unitarios tienen módulo unidad y son perpendiculares entre sí. En este sistema, los vectores unitarios son: $\vv[x]{u}, \vv[y]{u}, \vv[z]{u}$. 

El vector $\vv[x]{u}$ es perpendicular a la superficie coordenada $x = \text{cte.}$, y apunta en la dirección creciente de $x$. Análogamente para $\vv[y]{u}$ y $\vv[z]{u}$. Los vectores unitarios cartesianos mantienen la orientación relativa en todos los puntos del espacio.

El vector posición $\vv{r}$ se puede expresar como:
\begin{equation}
    \vv{r} = x \vv[x]{u} + y \vv[y]{u} + z \vv[z]{u}
\end{equation}

\subsubsection{Coordenadas cilíndricas}

Recomendables para problemas con \textbf{simetría cilíndrica}. Los valores que pueden tomar las coordenadas son:
\begin{equation}
    \begin{aligned}
        \rho &\in [0, +\infty) \\
        \varphi &\in [0, 2\pi) \\
        z &\in (-\infty, +\infty)
    \end{aligned}
\end{equation}

\begin{itemize}
    \item La coordenada $\rho$ es la distancia del punto al eje $z$.
    \item La coordenada $\varphi$ es el ángulo que forma el vector posición con el eje $x$.
    \item La coordenada $z$ es la distancia del punto al plano $xy$.
\end{itemize}

Los vectores unitarios cilíndricos son $\vv[\rho]{u}, \vv[\varphi]{u}, \vv[z]{u}$. El vector $\vv[\rho]{u}$ es perpendicular a la superficie coordenada $\rho = \text{cte.}$ (que es un cilindro), y apunta en la dirección creciente de $\rho$. El vector $\vv[\varphi]{u}$ es perpendicular a la superficie coordenada $\varphi = \text{cte.}$ (que es un plano perpendicular al eje $z$), y apunta en la dirección creciente de $\varphi$. El vector $\vv[z]{u}$ es perpendicular a la superficie coordenada $z = \text{cte.}$, y apunta en la dirección creciente de $z$.

El vector posición $\vv{r}$ se puede expresar como:
\begin{equation}
    \vv{r} = \rho \vv[\rho]{u} + z \vv[z]{u}
\end{equation}

\subsubsection{Coordenadas esféricas}

Recomendables para problemas con \textbf{simetría esférica}. Los valores que pueden tomar las coordenadas son:
\begin{equation}
    \begin{aligned}
        r &\in [0, +\infty) \\
        \theta &\in [0, \pi] \\
        \varphi &\in [0, 2\pi)
    \end{aligned}
\end{equation}

\begin{itemize}
    \item La coordenada $r$ es la distancia del punto al origen.
    \item La coordenada $\theta$ es el ángulo que forma el vector posición con el eje $z$.
    \item La coordenada $\varphi$ es el ángulo que forma el proyección del vector posición en el plano $xy$ con el eje $x$.
\end{itemize}

Los vectores unitarios esféricos son $\vv[r]{u}, \vv[\theta]{u}, \vv[\varphi]{u}$. El vector $\vv[r]{u}$ es perpendicular a la superficie coordenada $r = \text{cte.}$ (que es una esfera), y apunta en la dirección creciente de $r$. El vector $\vv[\theta]{u}$ es perpendicular a la superficie coordenada $\theta = \text{cte.}$ (que es un cono), y apunta en la dirección creciente de $\theta$. El vector $\vv[\varphi]{u}$ es perpendicular a la superficie coordenada $\varphi = \text{cte.}$ (que es un plano perpendicular al eje $z$), y apunta en la dirección creciente de $\varphi$.

El vector posición $\vv{r}$ se puede expresar como:
\begin{equation}
    \vv{r} = r \vv[r]{u}
\end{equation}

\subsection{Relación entre los sistemas de coordenadas}

\subsubsection{Cartesianas y cilíndricas}

\begin{equation}
    \left\lbrace 
    \begin{aligned}
        x &= \rho \cos \varphi \\
        y &= \rho \sin \varphi \\
        z &= z
    \end{aligned} \right. \qquad \qquad \left\lbrace 
    \begin{aligned}
        \rho &= \sqrt{x^2 + y^2} \\
        \varphi &= \arctan \left( \frac{y}{x} \right) \\
        z &= z
    \end{aligned} \right.
\end{equation}

\subsubsection{Cilíndricas y esféricas}

\begin{equation}
    \left\lbrace 
    \begin{aligned}
        \rho &= r \sin \theta \\
        \varphi &= \varphi \\
        z &= r \cos \theta
    \end{aligned} \right. \qquad \qquad \left\lbrace 
    \begin{aligned}
        r &= \sqrt{\rho^2 + z^2} \\
        \theta &= \arctan \left( \frac{\rho}{z} \right) \\
        \varphi &= \varphi
    \end{aligned} \right.
\end{equation}

\subsubsection{Cartesianas y esféricas}

\begin{equation}
    \left\lbrace 
    \begin{aligned}
        x &= r \sin \theta \cos \varphi \\
        y &= r \sin \theta \sin \varphi \\
        z &= r \cos \theta
    \end{aligned} \right. \qquad \qquad \left\lbrace 
    \begin{aligned}
        r &= \sqrt{x^2 + y^2 + z^2} \\
        \theta &= \arctan \left( \frac{\sqrt{x^2 + y^2}}{z} \right) \\
        \varphi &= \arctan \left( \frac{y}{x} \right)
    \end{aligned} \right.
\end{equation}

\subsection{Operador, campo escalar, campo vectorial}
Un operador es un ``objeto matemático'' que actúa sobre una función, transformándola. Sirve para estudiar las características de esa función.

Un ejemplo sería el operador \textbf{derivada}. Si se aplica el operador derivada a una función.

\begin{equation} \label{eq:ejemplo_operador_derivada}
    \dv{\sen \left( x \right)}{x} = g(x)
\end{equation}

Un campo escalar es una función que asigna un valor escalar a cada punto del espacio. Es decir,

\begin{equation} \label{eq:campo_escalar}
    f(x, y, z) \longrightarrow \mathbb{R} 
\end{equation}


\subsection{Gradiente}

El \textbf{gradiente} es un operador que se aplica a un campo escalar y devuelve un campo vectorial, de tal forma que en cada punto hay un vector que apunta en la dirección de mayor crecimiento del campo escalar. Se denota mediante el operador nabla: $\grad$.

\color{orange}
\begin{align} \label{eq:gradiente}
    \grad f &= \pdv{f}{x} \vv[x]{u} + \pdv{f}{y} \vv[y]{u} + \pdv{f}{z} \vv[z]{u}\\
    \grad f &= \pdv{f}{r} \vv[r]{u} + \frac{1}{r} \pdv{f}{\varphi} \vv[\varphi]{u} + \pdv{f}{z} \vv[z]{u} \\
    \grad f &= \pdv{f}{r} \vv[r]{u} + \frac{1}{r} \pdv{f}{\theta} \vv[\theta]{u} + \frac{1}{r \sin \theta} \pdv{f}{\varphi} \vv[\varphi]{u}
\end{align}
\color{black}

\subsubsection{Ejemplo}

\begin{align*}
    f \left( x,y,z \right) &= 3x^2 + y - \cos (z) \\
    \grad f &= \pdv{f}{x} \vv[x]{u} + \pdv{f}{y} \vv[y]{u} + \pdv{f}{z} \vv[z]{u} \\
     &= 6x \vv[x]{u} + \vv[y]{u} + \sin (z) \vv[z]{u}\\[10pt]
    f \left( 1,1,\frac{\pi}{6} \right) &= 3 + 1 - \frac{\sqrt{3}}{2} = \frac{7-\sqrt{3}}{2}\\
    \grad f \left( 1,1,\frac{\pi}{6} \right) &= 6 \vv[x]{u} + \vv[y]{u} + \frac{1}{2} \vv[z]{u}
\end{align*}

\subsection{Divergencia}
El \textbf{operador divergencia} se aplica a un campo vectorial y devuelve un campo escalar. Se denota también mediante el operador nabla ($\grad$, aunque en \LaTeX el comando \verb|\div| lo asigna como $\div$ porque utiliza el producto escalar entre el operador gradiente y el campo vectorial. Se define como:

\begin{equation} \label{eq:divergencia}
    \div \vv{a} = \text{div} \, \vv{a} = \lim _{V\to 0} \frac{1}{V} \oiint _S \vv{a} \cdot \dd{\vv{S}}
\end{equation}

Si la divergencia es positiva ($\div \vv{a} > 0$), el campo vectorial diverge o emana del punto (fuente). Si la divergencia es negativa ($\div \vv{a} < 0$), el campo vectorial converge o se dirige al punto (sumidero).

En lo diferentes sistemas de coordenadas, se puede calcular como:

\color{orange}
\begin{align}
    a
\end{align}
\color{black}



\subsubsection{Teorema de Gauss o de la divergencia}

La superficie puede ser cualquier superficie cerrada, no necesariamente una esfera.

\begin{equation} \label{eq:teorema_gauss}
    \iiint _V \div \vv{a} \dd{V} = \oiint _S \vv{a} \cdot \dd{\vv{S}}
\end{equation}

El flujo de un campo vectorial $\vv{a}$ a través de una superficie cerrada equivale a la integral del volumen de la divergencia de $\vv{a}$.

\subsection{Rotacional}

El rotacional se aplica a un campo vectorial y devuelve un campo vectorial. Se representa también por el operador nabla ($\grad$), aunque en \LaTeX el comando \verb|\curl| lo asigna como $\curl$. 

En los diferentes sistemas de coordenadas, se puede calcular como:
\color{orange}
\begin{align}
    % Cartesianas
    \curl \vv{a} &= \left( \pdv{a_z}{y} - \pdv{a_y}{z} \right) \vv[x]{u} + \left( \pdv{a_x}{z} - \pdv{a_z}{x} \right) \vv[y]{u} + \left( \pdv{a_y}{x} - \pdv{a_x}{y} \right) \vv[z]{u} \\
    % Cilíndricas
    \curl \vv{a} &= \frac{1}{\rho} \left( \pdv{a_z}{\rho} - \pdv{a_\rho}{z} \right) \vv[\rho]{u} + \frac{1}{\rho} \left( \pdv{a_\rho}{z} - \pdv{a_z}{\rho} \right) \vv[\varphi]{u} + \pdv{a_\varphi}{\rho} \vv[z]{u} \\
    % Esféricas
    \begin{split}
        \curl \vv{a} &= \frac{1}{r \sin \theta} \left( \pdv{a_\varphi}{\theta} - \pdv{a_\theta}{\varphi} \right) \vv[r]{u} + \frac{1}{r} \left( \frac{1}{\sin \theta} \pdv{a_r}{\varphi} - \pdv{a_\varphi}{r} \right) \vv[\theta]{u}\\    &+ \frac{1}{r} \left( \pdv{a_r}{\theta} - \pdv{a_\theta}{r} \right) \vv[\varphi]{u}
    \end{split}
\end{align}
\color{black}

Un campo es \textbf{irrotacional} si su rotacional es nulo.
    \begin{equation} \label{eq:campo_irrotacional}
        \curl \vv{a} = 0
    \end{equation}

La condición necesaria para que un campo vectorial $\vv{a}$ pueda expresarse como el gradiente de un campo escalar $\phi$ es que su rotacional sea nulo. 
    
    \begin{equation} \label{eq:funcion_potencial}
        \curl \vv{a} = 0 \quad \Longleftrightarrow \quad \vv{a} = \grad \phi
    \end{equation} 

\subsection{Teorema de Stokes}

Se escoge una curva cerrada $C$ y una superficie $S$ de forma arbitraria que contenga en su contorno la curva $C$. Entonces, se cumple que el flujo del vector rotacional de un campo vectorial $\vv{a}$ a través de la superficie $S$ es igual a la circulación de $\vv{a}$ a lo largo de la curva $C$.

\begin{equation} \label{eq:teorema_stokes}
    \color{orange}
    \iint _S \left( \curl \vv{a} \right) \cdot \dd{\vv{S}} = \oint _C \vv{a} \cdot \dd{\vv{r}}
\end{equation}

Para campos conservativos, el flujo del rotacional a través de una superficie cerrada es nulo. Por ello, este teorema puede ser útil para resolver 

TODO Repetir el 7 de las tareas de clase (hice mal un vector unitario).

\subsection{Laplaciano}

El \textbf{operador laplaciano} se aplica a un campo escalar y devuelve un campo escalar. Se denota mediante el operador nabla ($\grad ^2$), aunque en \LaTeX el comando \verb|\laplacian| lo asigna como $\laplacian$ o a veces incluso como un triángulo ($\Delta$).

\begin{equation} \label{eq:laplaciano}
    \Delta f = \laplacian f = \div \left( \grad f \right)
\end{equation}

En coordenadas cartesianas, cilíndricas y esféricas, se puede calcular como:

\color{orange}
\begin{align}
    % Cartesianas
    \laplacian f &= \pdv[2]{f}{x} + \pdv[2]{f}{y} + \pdv[2]{f}{z} \\
    % Cilíndricas
    \laplacian f &= \frac{1}{\rho} \pdv{\rho} \left( \rho \pdv{f}{\rho} \right) + \frac{1}{\rho^2} \pdv[2]{f}{\varphi} + \pdv[2]{f}{z} \\
    % Esféricas
    \laplacian f &= \frac{1}{r^2} \pdv{r} \left( r^2 \pdv{f}{r} \right) + \frac{1}{r^2 \sin \theta} \pdv{\theta} \left( \sin \theta \pdv{f}{\theta} \right) + \frac{1}{r^2 \sin^2 \theta} \pdv[2]{f}{\varphi}
\end{align}
\color{black}

Las ecuaciones de onda están escritas en función de laplacianos, por ello son relevantes.

\subsection{Cálculo de la función potencial}

Lo realizaremos con un ejemplo. Dado el campo vectorial:
\[ \vv{a} = y \vv[x]{u} + \left( 2yz + x \right) \vv[y]{u} + y^2 \vv[z]{u} \]
Y sabiendo que se trata del gradiente de un campo escalar, calcular su función potencial $\phi$.

\[
\begin{split}
    \left. \begin{aligned}
        \grad \phi &= \pdv{\phi}{x} \vv[x]{u} + \pdv{\phi}{y} \vv[y]{u} + \pdv{\phi}{z} \vv[z]{u} \\
        &= a_x \vv[x]{u} + a_y \vv[y]{u} + a_z \vv[z]{u}
    \end{aligned}
     \right\rbrace \Rightarrow
     \left\lbrace \begin{aligned}
        \pdv{\phi}{x} &= y \\
        \pdv{\phi}{y} &= 2yz + x \\
        \pdv{\phi}{z} &= y^2
     \end{aligned} \right. \dots \\
     \Rightarrow
        \left\lbrace \begin{aligned}
            \phi &= \int y \dd{x} &= xy + f_1(y,z) \\
            \phi &= \int \left( 2yz + x \right) \dd{y} &= 2y^2z + xy + f_2(x,z) \\
            \phi &= \int y^2 \dd{z} &= y^2z + f_3(x,y)
        \end{aligned} \right.
\end{split}
\]

TODO Problema 5 de las tareas de clase

\subsection{Campo solenoidal}

Un campo vectorial es \textbf{solenoidal} si su divergencia es nula. Es decir:
\begin{equation} \label{eq:campo_solenoidal}
    \div \vv{a} = 0
\end{equation}

Si un campo es solenoidal, este se puede expresar como el rotacional de otro campo vectorial, al que se le llama \textbf{potencial vector}.

\begin{equation} \label{eq:potencial_vector}
    \div \vv{a} = 0 \quad \Longleftrightarrow \quad \vv{a} = \curl \vv{A}
\end{equation}

TODO Comprobar cálculo de la función potencial
TODO Problemas 6 y 11 de las tareas de clase.
TODO Comprobar que los ejercicios anteriores están bien (ver fotos)

\section{Ondas acústicas planas}

\subsection{Notación compleja}

Una función de onda general se expresa en notación armónica como:

\begin{equation} \label{eq:onda_general}
    \Psi (x,t) = \Psi _0 \cos (\omega t -kx + \varphi)
\end{equation}

La notación armónica, que utiliza las funciones seno y coseno, dificulta cierto tipo de operaciones matemáticas. Por ello, se busca 

\[ \Psi _0 e^{i \left( \omega t - kx + \varphi \right)} = \Psi _0 \left[ \cos \left( \omega t - kx + \varphi \right) + i \sen \left( \omega t - kx + \varphi \right) \right] \]

La notación exponencial utiliza la ecuación de Euler, que relaciona las funciones trigonométricas con las funciones exponenciales, \textbf{ignorando} que estamos cogiendo la parte real. 
\begin{equation} \label{eq:notacion_compleja}
    \Psi (x,t) = \Re \left\lbrace \Psi _0 e^{i \left( \omega t - kx + \varphi \right)} \right\rbrace \quad \Longrightarrow \quad \Psi (x,t) =  \Psi _0 e^{i \omega t - kx + \varphi}
\end{equation}

Las operaciones como derivación e integración se simplifican con la notación compleja.

\begin{equation} \label{eq:derivacion_notacion_compleja}
    \frac{\dd ^n t}{\dd t ^n} = \left( i \omega \right) ^n \Psi (x, t)
\end{equation}

\begin{equation} \label{eq:integracion_notacion_compleja}
    \int \Psi (x,t) \dd{t} = \frac{1}{i \omega}\Psi (x,t) + C
\end{equation}

En el plano complejo, un número complejo $z$ se expresa como:
\[ z = a + ib = r e^{i \varphi} \]

Donde $a$ es la parte real, $b$ es la parte imaginaria, $r$ es el módulo y $\varphi$ es el argumento. La relación entre estos parámetros es la siguiente:

\begin{align*}
    r &= \sqrt{a^2 + b^2} & a &= r \cos \varphi \\
    \varphi &= \arctan \left( \frac{b}{a} \right) & b &= r \sen \varphi
\end{align*}

Si tenemos una función de onda tal que $\Psi (x,t) = 2i e^{i \left( \omega t - kx + \varphi \right)}$, deberemos transformarlo a notación compleja de la siguiente forma:

\[ \Psi (x,t) = 2i e^{i \left( \omega t - kx + \varphi \right)} =2 e^{i \frac{\pi}{2} } e^{i \left( \omega t - kx + \varphi \right)} = 2 e^{i \left( \omega t - kx + \varphi + \frac{\pi}{2} \right)} \]

Por otro lado, si tenemos una función de onda tal que $\Psi (x,t) = a e^{i \left( \omega t - kx \right)} + b e^{i \left( \omega t + kx + \varphi \right)}$, deberemos transformarlo a notación armónica.

\subsection{Acústica lineal}

El rango de frecuencias audible es de \qty{20}{\Hz} a \qty{20}{\kHz}. Si la frecuencia de una onda acústica es menor a \qty{20}{\Hz} o mayor a \qty{20}{\kHz}, se denominará \textbf{infrasónica} o \textbf{ultrasónica}, respectivamente.

Las ondas acústicas son ondas mecánicas. Esto quiere decir que necesitan de un medio para transmitirse. Entendiendo el aire como medio, cada partícula de este fluido se situa en equilibrio en un punto fijo $x_0$ efectuando un Movimiento Armónico Simple (MAS) alrededor de este punto.

TODO Añadir paquete para términos de glosario.

\begin{align} \label{eq:presion_acustica}
    p(\vv{r}, t) &= p_0 e^{i \left( \omega t - \vv{k} \cdot \vv{r} + \varphi\right)}\\
    \vv{\xi} (\vv{r}, t) &= \vv{\xi} _0 e^{i \left( \omega t - \vv{k} \cdot \vv{r} + \varphi - \frac{\pi}{2}\right) }\vv[k]{u}\\
    \vv[p]{v} (\vv{r}, t) &= \vv{v} _0 e^{i \left( \omega t - \vv{k} \cdot \vv{r} + \varphi\right) }\vv[k]{u}
\end{align}

Donde:
\begin{itemize}
    \item $\xi$ es el desplazamiento de la partícula con respecto a su posición de equilibrio.
    \item $\vv[p]{v}$ es la velocidad vibratoria de la partícula.
\end{itemize}

Función de onda genérica que puede representar cualquiera de las tres anteriores.

\[ \Psi (\vv{r}, t) = \Psi _0 e^{i \left( \omega t - \vv{k} \cdot \vv{r} + \varphi\right)} \]

Donde:
\begin{align*}
    \Psi _0 &\equiv \text{Amplitud de la onda, y es \textbf{constante para ondas planas}.} \\
    \omega &\equiv \text{Frecuencia angular.} \\
    \vv{k} &\equiv \text{Vector de onda.} \\
    \vv{r} &\equiv \text{Vector posición.} \\
\end{align*}

Los vectores $\vv{k}$ y $\vv{r}$ representan:
\begin{align*}
    \vv{k} & = k_x \vv[x]{u} + k_y \vv[y]{u} + k_z \vv[z]{u} \\
    \vv{r} & = x \vv[x]{u} + y \vv[y]{u} + z \vv[z]{u}
\end{align*}

Por lo tanto, el producto escalar que aparece en el exponente de la función de onda será:
\[ \vv{k} \cdot \vv{r} = k_x x + k_y y + k_z z \]

Por ejemplo, siendo una función de onda $\Psi (x,t) = 3 e^{i \left( 3000t - 2x + 3y + \frac{\pi}{3}\right)}$, sus parámetros son:
\begin{itemize}
    \item Amplitud: $ \Psi_0 = 3$
    \item Frecuencia angular: $\omega = 3000$
    \item Vector de onda: $\vv{k} = -2 \vv[x]{u} + 3 \vv[y]{u}$
    \item Fase: $\varphi = \frac{\pi}{3}$
\end{itemize}

TODO Ejercicio: Identificar parámetros de las siguientes funciones de onda.

\begin{itemize}
    \item $\Psi(\vv{r, t}) = - \Psi_0 e^{i \left( \omega t - kz + \varphi \right)}$
    \item $\Psi(\vv{r, t}) = (1-i) \Psi_0 e^{i \left( \omega t - kx + \varphi \right)}$
    \item $\Psi(\vv{r, t}) = i \Psi_0 e^{i \left( \omega t - ky + \varphi \right)}$
\end{itemize}

Las ondas acústicas son un tipo de \textbf{onda longitudinal}. Esto quiere decir que la dirección de propagación de la onda es la misma que la dirección de oscilación de las partículas del medio. 

La \textbf{ecuación de estado} dice:
\begin{equation} \label{eq:ecuacion_estado}
    p = B S
\end{equation}

La ecuación de continuidad dice:

La ecuación de Euler es muy importante:

\begin{equation} \label{eq:ecuacion_euler}
    \color{orange}
    \grad p = - \rho_0 \pdv{\vv{v_p}}{t}
\end{equation}

Donde $\rho_0$ es la densidad de equilibrio del fluido, y $\vv{v_p}$ es la velocidad de las partículas del fluido.

\[ \laplacian p = \frac{\rho_0}{B} \pdv[2]{p}{t} \]

La velocidad de propagación de la onda se obtiene como:

\begin{equation} \label{eq:velocidad_del_medio}
    \vv{v_s} = \sqrt{\frac{B}{\rho_0}}
\end{equation}

Por lo tanto, la velocidad de propagación de la onda acústica \textbf{depende el medio}. Esto es importante al relacionar la velocidad de propagación con la longitud de onda y la frecuencia:

\begin{equation} \label{eq:velocidad_longitud_frecuencia}
    v_s = \lambda f = \frac{\omega }{k}
\end{equation}

Si el medio no varía, $v$ es constante. Por lo tanto, si la longitud de onda aumenta, la frecuencia disminuye y viceversa.

\begin{align*}
    p(\vv{r}, t) &= p_0 \cos \left( \omega t - \vv{k} \cdot \vv{r} + \varphi \right)\\ 
    &= p_0 e^{i \left( \omega t - \vv{k} \cdot \vv{r} + \varphi \right)}
\end{align*}

La \textbf{longitud de onda} $\lambda$ es la distancia entre dos puntos de la onda que están en fase. Se relaciona con el vector de onda $\vv{k}$ como:

\begin{equation} \label{eq:longitud_onda}
    k = \abs{\vv{k}} = \frac{2\pi}{\lambda}
\end{equation}


Importante (cae en exámenes). Partiendo de la función de onda de la presión acústica, podemos llegar a la función de onda de la velocidad de las partículas mediante la ecuación de Euler. 

\begin{align*}
    \grad p &= - \rho_0 \pdv{\vv{v_p}}{t}\\
    - \frac{1}{\rho_0} \grad p &= \pdv{\vv{v_p}}{t}\\
    \vv[p]{v} = - \frac{1}{\rho_0} \int \grad p \dd{t} &= - \frac{1}{\rho_0} \int \grad p (\vv{r}, t) \dd{t}
\end{align*}

Usaremos coordenadas cartesianas porque se trata de ondas planas. Recordamos que para simetrías planas las coordenadas cartesianas facilitan los cálculos.

\begin{align*}
    p(\vv{r}, t) &= p_0 e^{i \left( \omega t - \vv{k} \cdot \vv{r} + \varphi \right)}\\
    p(\vv{r}, t) &= p_0 e^{i \left( \omega t - k_x x - k_y y - k_z z + \varphi \right)}\\[15pt]
    \grad p &= \pdv{p}{x} \vv[x]{u} + \pdv{p}{y} \vv[y]{u} + \pdv{p}{z} \vv[z]{u} \\[15pt]
    \pdv{p_x}{x} &= \pdv{}{x} \left\lbrace p_0 e^{i\left(  \omega t - k_x x - k_y y - k_z z + \varphi \right)} \right\rbrace = - i k_x p_0 \sen \left( \omega t - kx + \varphi \right)
\end{align*}

TODO Continuar donde aparece la marca $\boxed{\ast\triangle}$

\begin{align*}
    \grad p &= - i k_x p_0 e ^{i \left( \omega t - \vv{k} \cdot \vv{r} + \varphi \right)} \\ 
    \vv[p]{v} (\vv{r}, t) &= \frac{i p_0 \vv{k}}{\rho_0} \int_{0}^{t} e ^{i \left( \omega t ' - \vv{k} \cdot \vv{r} + \varphi \right)} \dd{t'}
\end{align*}

La variable $t'$ es una variable muda que se utiliza para distinguir la variable de integración de los límites de integración.

\begin{align*}
    \vv[p]{v} (\vv{r}, t) &= \frac{i p_0 \vv{k}}{\rho_0} \cdot \frac{1}{i \omega} \biggl. e ^{i \left( \omega t - \vv{k} \cdot \vv{r} + \varphi \right)}  \biggr\vert _{0}^{t}\\ 
    &= \frac{\vv{k}}{\rho_0 \omega} p_0 e ^{i \left( \omega t - \vv{k} \cdot \vv{r} + \varphi - \frac{\pi}{2} \right)}\\
\end{align*}

El vector de onda $\vv{k}$ se expresa como:
\[ \vv{k} = k_x \vv[x]{u} + k_y \vv[y]{u} + k_z \vv[z]{u} = \abs{\vv{k}} \vv[k]{u} \]

Por lo tanto, la direcciń de propagación de la onda será:
\[ \vv[k]{u} = \frac{\vv{k}}{\abs{\vv{k}}} = \frac{k_x \vv[x]{u} + k_y \vv[y]{u} + k_z \vv[z]{u}}{\abs{\vv{k}}} \]

Las amplitudes $p_0, \xi_0, v_0$ son constantes. La presión y la velocidad están en fase, mientras que el desplazamiento está retrasado $\frac{\pi}{2}$.


\subsection{Ecuación de onda. Solución armónica}

TODO Averiguar qué va en esta parte de todo lo que he copiado


\subsection{Densidad de energía. Intensidad acústica}

La intensidad acústica se define como la  cantidad de energía que fluye por unidad de tiempo y por unidad de superficie perpendicular a la dirección de propagación de la onda.
\begin{equation} \label{eq:intensidad_acustica}
    I = \frac{P}{S} \qquad \left[ I \right] = \si{\watt\per\meter\squared}
\end{equation}

La intensidad instantánea se define como el producto de la presión acústica por la velocidad de las partículas del medio.
\begin{equation} \label{eq:intensidad_acustica_instantanea}
    I _{\text{ins}} = p \cdot v_p
\end{equation}

Si desarrollamos:
\begin{align*}
    I _{\text{ins}} = p \cdot v_p &= p_0 e^{i \left( \omega t - \vv{k} \cdot \vv{r} + \varphi \right)} \cdot v_0 e^{i \left( \omega t - \vv{k} \cdot \vv{r} + \varphi \right)} \\ 
    &= p_0 \cos \left( \omega t - \vv{k} \cdot \vv{r} + \varphi \right) \cdot v_0 \cos \left( \omega t - \vv{k} \cdot \vv{r} + \varphi \right) \\
\end{align*}

La \textbf{intensidad de la onda acústica} se define como el valor medio de la intensidad instantánea.

\begin{equation} \label{eq:intensidad_onda}
    \begin{split}
        I = \left\langle I _{\text{ins}} \right\rangle = \left\langle p \cdot v_p \right\rangle = \frac{1}{T} \int_{0}^{T} p \cdot v_p \dd{t}  \\= \frac{1}{T} \int_{0}^{T} p_0 v_0 \cos^2 \left( \omega t - \vv{k} \cdot \vv{r} + \varphi \right) \dd{t} \\= \frac{\rho_0 v_0}{T} \int_{0}^{T} \frac{1 + \cancelto{0}{\cos \left[ 2 \left( \omega t - \vv{k} \cdot \vv{r} + \varphi \right) \right]} }{2} \dd{t} \\= \frac{\rho_0 v_0}{T} \int_{0}^{T} \frac{1}{2} \dd{t} = \frac{\rho_0 v_0}{2}
    \end{split}
\end{equation}

Por lo tanto,

\begin{equation} \label{eq:intensidad_onda_final}
    I = \frac{1}{2} \rho_0 v_s \omega ^2 \xi _0 ^2
\end{equation}

\[ S = 10 \log \left( \frac{I}{I_0}  \right) \unit{\dB}  \]

Donde $I_0$ es la intensidad acústica de referencia, que corresponde al umbral de audición del ser humano, y su valor es $I_0 = \qty{e-12}{\watt\per \meter\squared}$.

\[ S_1 - S_2 = 10 \log \left( \frac{I_1}{I_2} \right) \unit{\dB} \]

La \textbf{impedancia característica del medio} se define como:
\begin{equation} \label{eq:impedancia_caracteristica}
    Z_m = \rho_0 v_s
\end{equation}

La \textbf{impedancia aćustica} es el cociente entre la presión acústica y la velocidad de las partículas del medio.
\begin{equation} \label{eq:impedancia_acustica} \color{orange}
    Z = \frac{p}{v_p}
\end{equation}

Las unidades de la impedancia acústica y de la impedancia característica del medio son los Rayleighs (\unit{\rayl}).

\begin{equation} \label{eq:interferencias}
    I_T(P) = I_1 + I_2 + 2 \sqrt{I_1I_2}\cos \left[ \delta (P) \right]
\end{equation}

El valor máximo sucede cuando el coseno es 1, por lo tanto el valor máximo de la intensidad total es:
\[ I_{\text{max}}(P) = \left( \sqrt{I_1} + \sqrt{I_2} \right) ^2 \]

TODO: Problemas 2.1 y 2.4

\section{Ondas acústicas esféricas}
\section{Reflexión y refracción de ondas acústicas}
\section{Ecuaciones de Maxwell. Ecuaciones de onda. Energía}
\section{Propagación de ondas electromagnéticas en medios dieléctricos}
\section{Propagación de ondas electromagnéticas en medios conductores}
\section{Reflexión y refracción de ondas electromagnéticas}
\section{Ondas guiadas}

\section{Soluciones de los ejercicios}

\subsection{Diapositiva 75 Tema 1, mayo 2020}

COMPROBAR

Sea el campo vectorial,

\[ \vv{a} = \left[ 8x^2 \sen \varphi  + \left( 16xy-z \right) \cos \varphi \right] \vv[\rho]{u} + \left[ 8x^2 \cos \varphi  - \left( 16xy-z \right) \sen \varphi \right] \vv[\varphi]{u} - x \vv[z]{u} \]

\begin{enumerate}
     \item Determinar si es irrotacional o no. \textbf{Solución:} Sí es irrotacional.
     \item Calcular el flujo neto de $\vv{a}$ a través del cubo $0<x,y,z<1$. \textbf{Solución:} Usando Teorema de Gauss \autoref{eq:teorema_gauss} el resultado es 8.
     \item Determinar la circulación de $\vv{a}$ a lo largo del perímetro de la cara del cubo definida por $z=0$ y por $0<x,y<1$. \textbf{Solución:} Usando Teorema de Stokes \autoref{eq:teorema_stokes}, el resultado es 0.
\end{enumerate}



\end{document}